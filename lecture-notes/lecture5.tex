\PassOptionsToPackage{unicode=true}{hyperref} % options for packages loaded elsewhere
\PassOptionsToPackage{hyphens}{url}
%
\documentclass[]{article}
\usepackage{lmodern}
\usepackage{amssymb,amsmath}
\usepackage{ifxetex,ifluatex}
\usepackage{fixltx2e} % provides \textsubscript
\ifnum 0\ifxetex 1\fi\ifluatex 1\fi=0 % if pdftex
  \usepackage[T1]{fontenc}
  \usepackage[utf8]{inputenc}
  \usepackage{textcomp} % provides euro and other symbols
\else % if luatex or xelatex
  \usepackage{unicode-math}
  \defaultfontfeatures{Ligatures=TeX,Scale=MatchLowercase}
\fi
% use upquote if available, for straight quotes in verbatim environments
\IfFileExists{upquote.sty}{\usepackage{upquote}}{}
% use microtype if available
\IfFileExists{microtype.sty}{%
\usepackage[]{microtype}
\UseMicrotypeSet[protrusion]{basicmath} % disable protrusion for tt fonts
}{}
\IfFileExists{parskip.sty}{%
\usepackage{parskip}
}{% else
\setlength{\parindent}{0pt}
\setlength{\parskip}{6pt plus 2pt minus 1pt}
}
\usepackage{hyperref}
\hypersetup{
            pdfborder={0 0 0},
            breaklinks=true}
\urlstyle{same}  % don't use monospace font for urls
\setlength{\emergencystretch}{3em}  % prevent overfull lines
\providecommand{\tightlist}{%
  \setlength{\itemsep}{0pt}\setlength{\parskip}{0pt}}
\setcounter{secnumdepth}{0}
% Redefines (sub)paragraphs to behave more like sections
\ifx\paragraph\undefined\else
\let\oldparagraph\paragraph
\renewcommand{\paragraph}[1]{\oldparagraph{#1}\mbox{}}
\fi
\ifx\subparagraph\undefined\else
\let\oldsubparagraph\subparagraph
\renewcommand{\subparagraph}[1]{\oldsubparagraph{#1}\mbox{}}
\fi

% set default figure placement to htbp
\makeatletter
\def\fps@figure{htbp}
\makeatother


\date{}

\begin{document}

\hypertarget{verification-cycle-verification-methodology-verification-plan}{%
\section{Verification Cycle, Verification Methodology \& Verification
Plan}\label{verification-cycle-verification-methodology-verification-plan}}

Functional specification -\textgreater{} Designer implements the
functional specification (in HDL) -\textgreater{} create verification
plan -\textgreater{} develop verification environment -\textgreater{}
debugging -\textgreater{} run regression -\textgreater{} tape out
readiness -\textgreater{} debug fabricated hardware -\textgreater{}
escape analysis

\hypertarget{create-verification-plan}{%
\subsection{Create verification Plan}\label{create-verification-plan}}

\begin{itemize}
\tightlist
\item
  functions to be verified
\item
  resources required and schedule details
\item
  required tools
\item
  specific tests and methods
\item
  completion criteria - how do you measure success?
\end{itemize}

\hypertarget{develop-verification-environment}{%
\subsection{Develop verification
environment}\label{develop-verification-environment}}

\begin{itemize}
\tightlist
\item
  set of software code and tools that enable the verification engineer
  to identify flaws in the design
\item
  stimulus and checking for stimulation based environments
\item
  rules generation (properties) for formal verification environments
\end{itemize}

\hypertarget{debug-hdl-and-environment}{%
\subsection{Debug HDL and environment}\label{debug-hdl-and-environment}}

\begin{itemize}
\tightlist
\item
  run tests according to the verification plan and look for anomalies
\item
  examine the anomalies to reveal failure source
\item
  fix cause of the failure
\item
  rerun the same test
\item
  update verification plan based on the test results
\end{itemize}

\hypertarget{run-regression}{%
\subsection{Run regression}\label{run-regression}}

\begin{itemize}
\tightlist
\item
  continuous running of the tests defined in the verification plan
\item
  simulation farms
\item
  used to uncover hard-to-find bugs and ensure that the quality of the
  design keeps improving
\end{itemize}

\hypertarget{debug-fabricated-hardware}{%
\subsection{Debug fabricated hardware}\label{debug-fabricated-hardware}}

\begin{itemize}
\tightlist
\item
  hardware bring-up
\end{itemize}

\hypertarget{perform-escape-analysis}{%
\subsection{Perform escape analysis}\label{perform-escape-analysis}}

\begin{itemize}
\tightlist
\item
  reproduce the bug in a simulation environment, if possible
\item
  fully understand the bug -\textgreater{} reasons why it went
  undiscovered by the verification environments
\end{itemize}

\hypertarget{verification-methodology}{%
\subsection{Verification methodology}\label{verification-methodology}}

\hypertarget{test-patterns}{%
\subsubsection{Test patterns}\label{test-patterns}}

\begin{itemize}
\tightlist
\item
  patterns created to test specific behaviours
\item
  each pattern handles a single scenario
\item
  hand generated
\item
  DUV behaviour is manually checked
\item
  bare expensive
\end{itemize}

\hypertarget{test-cases}{%
\subsubsection{Test cases}\label{test-cases}}

\begin{itemize}
\tightlist
\item
  mostly change the checking of the test
\item
  checking evolves to self checking tests/automatic checking (checking
  is independent of the stimulus)
\end{itemize}

Test patterns -\textgreater{} test cases -\textgreater{} test generators
-\textgreater{} test drivers

\hypertarget{test-case-generators}{%
\subsubsection{Test case generators}\label{test-case-generators}}

\begin{itemize}
\tightlist
\item
  machine generated random patterns
\item
  more generic targets, multiple scenarios and large number of tests
\end{itemize}

\hypertarget{test-case-drivers}{%
\subsubsection{Test case drivers}\label{test-case-drivers}}

\begin{itemize}
\tightlist
\item
  stimuli generation is embedded in the verification environment
\item
  stimuli are generated the operation of the environment (and
  stimulation)
\item
  driver can react to the state of the DUV
\end{itemize}

\hypertarget{evolution-of-the-verification-plan}{%
\subsection{Evolution of the verification
Plan}\label{evolution-of-the-verification-plan}}

\begin{itemize}
\tightlist
\item
  functional specification document
\item
  understand the DUV before determining how to verify it
\item
  confront unclear and ambiguous definitions
\item
  incomplete and changing continuously
\end{itemize}

\hypertarget{verification-plan}{%
\subsection{Verification Plan}\label{verification-plan}}

\begin{itemize}
\tightlist
\item
  verification levels -- which levels to perform the verification --
  complexity, resources, risk, existence of a clean interface and spec
\item
  functions -- specific functions of the DUV that the verification team
  will exercise -- assign priority for each function --- critical
  functions --- secondary functions -- functions not verified at this
  level --- fully verified at a lower level --- not applicable to this
  level -- required tools -- specific tests and methods --- type of
  verification: white, black, grey --- verification strategy: formal
  verification, deterministic, random based --- abstraction level: from
  bit-level to algorithmic level from bit-streams to transactions ---
  checking: simple IO checking for data correctness, behavioural rules
  for timing -- abstraction levels --- bit-level -\textgreater{} command
  \& data ``packet'' level -\textgreater{} sequence level
  -\textgreater{} program or algorithmic level -- coverage requirements
  --- feedback mechanism that evaluates the quality of the stimuli ---
  required in all random-based verification environments --- defined as
  events -- completion criteria
\end{itemize}

\end{document}
